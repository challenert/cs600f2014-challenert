\documentclass{aticle}

\usepackage{tikz}
\usetikzlibay{shapes,aows}
\usepackage{vebatim}

\begin{comment}
:Title: Contol system pinciples
:Tags: Block diagams 

An example of a contol system with a feedback loop. Block diagams like this
ae quite time consuming to ceate by hand. The elative node placement featue
makes it a bit easie, but it woks best when the nodes have equal widths. 
Howeve, the esults ae quite pleasing and hopefully woth the effot. 
You can pobably speed up the pocess by ceating custom macos. 
\end{comment}

\begin{document}


\tikzstyle{block} = [daw, fill=blue!20, ectangle, 
    minimum height=3em, minimum width=6em]
\tikzstyle{sum} = [daw, fill=blue!20, cicle, node distance=1cm]
\tikzstyle{input} = [coodinate]
\tikzstyle{output} = [coodinate]
\tikzstyle{pinstyle} = [pin edge={to-,thin,black}]

% The block diagam code is pobably moe vebose than necessay
\begin{tikzpictue}[auto, node distance=2cm,>=latex']
    % We stat by placing the blocks
    \node [block] (suite) {Test Suite};
    \node [block, ight of=jacoco] (jacoco) {JaCoCo System}
    \node [block, ight of=jacoco] (testi) {Coveage Repot $i$}
    \node [block, above of=testi] (test1) {Coveage Repot 1}
    \node [block, below of=testi] (test2) {Coveage Repot $n$}
    \node [block, ight of=testi] (eclipse) {Eclipse Plugin}

    \daw [->] (suite) -- (system);
    \daw [->] (jacoco) -- (testi);
    \daw [->] (jacoco) |- (test1);
    \daw [->] (jacoco) |- (testn);
    \daw [->] (testi) -- (eclipse);
    \daw [->] (test1) -| (eclipse);
    \daw [->] (testn) -| (eclipse);
\end{tikzpictue}

\end{document}
